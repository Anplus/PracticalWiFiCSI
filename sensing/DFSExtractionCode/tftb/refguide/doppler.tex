% This is part of the TFTB Reference Manual.
% Copyright (C) 1996 CNRS (France) and Rice University (US).
% See the file refguide.tex for copying conditions.



\markright{doppler}
\section*{\hspace*{-1.6cm} doppler}

\vspace*{-.4cm}
\hspace*{-1.6cm}\rule[0in]{16.5cm}{.02cm}
\vspace*{.2cm}



{\bf \large \sf Purpose}\\
\hspace*{1.5cm}
\begin{minipage}[t]{13.5cm}
Complex Doppler signal.
\end{minipage}
\vspace*{.5cm}


{\bf \large \sf Synopsis}\\
\hspace*{1.5cm}
\begin{minipage}[t]{13.5cm}
\begin{verbatim}
[fm,am,iflaw] = doppler(N,fs,f0,d,v) 
[fm,am,iflaw] = doppler(N,fs,f0,d,v,t0) 
[fm,am,iflaw] = doppler(N,fs,f0,d,v,t0,c) 
\end{verbatim}
\end{minipage}
\vspace*{.5cm}


{\bf \large \sf Description}\\
\hspace*{1.5cm}
\begin{minipage}[t]{13.5cm}
         {\ty doppler} returns the frequency modulation ({\ty fm}), the
         amplitude modulation ({\ty am}) and the instantaneous frequency
         law ({\ty iflaw}) of the signal received by a fixed observer from
         a moving target emitting a pure frequency {\ty f0}.\\
 
\hspace*{-.5cm}\begin{tabular*}{14cm}{p{1.5cm} p{8.5cm} c}
Name & Description & Default value\\
\hline
         {\ty N}  & number of points\\
         {\ty fs} & sampling frequency (in Hz)\\
         {\ty f0} & target   frequency (in Hz)\\
         {\ty d}  & distance from the line to the observer (in meters)\\
         {\ty v}  & target velocity    (in m/s)\\
         {\ty t0} & time center                  & {\ty N/2}\\  
         {\ty c}  & wave velocity      (in m/s)  & {\ty 340}\\
 \hline  {\ty fm} & output frequency modulation\\  
         {\ty am} & output amplitude modulation\\  
         {\ty iflaw} & output instantaneous frequency law\\

\hline
\end{tabular*}
\vspace*{.2cm}

The doppler effect characterizes the fact that a signal returned from a
moving target is scaled and delayed compared to the transmitted signal. For
narrow-band signals, this scaling effect can be considered as a frequency
shift. \\

{\ty [fm,am,iflaw] = doppler(N,fs,f0,d,v,t0,c)} returns the signal received
by a fixed observer from a moving target emitting a pure frequency {\ty
f0}. The target is moving along a straight line, which gets closer to the
observer up to a distance {\ty d}, and then moves away. {\ty t0} is the
time center (i.e. the time at which the target is at the closest distance
from the observer), and {\ty c} is the wave velocity in the medium.

\end{minipage}

\newpage

{\bf \large \sf Example}\\
\hspace*{1.5cm}
\begin{minipage}[t]{13.5cm}
Plot the signal and its instantaneous frequency law received by an observer
from a car moving at the speed $v=50 m/s$, passing at 10 meters from the
observer (the radar). The rotating frequency of the engine is $f_0=65 Hz$,
and the sampling frequency is $f_s=200 Hz$ :
\begin{verbatim}
     N=512; [fm,am,iflaw]=doppler(N,200,65,10,50); 
     subplot(211); plot(real(am.*fm)); 
     subplot(212); plot(iflaw);
     [ifhat,t]=instfreq(sigmerge(am.*fm,noisecg(N),15),11:502,10);
     hold on; plot(t,ifhat,'g'); hold off;
\end{verbatim}
\end{minipage}
\vspace*{.5cm}


{\bf \large \sf See Also}\\
\hspace*{1.5cm}
\begin{minipage}[t]{13.5cm}
\begin{verbatim}
dopnoise.
\end{verbatim}
\end{minipage}

