% This is part of the TFTB Reference Manual.
% Copyright (C) 1996 CNRS (France) and Rice University (US).
% See the file refguide.tex for copying conditions.


\markright{tftb\_window}
\section*{\hspace*{-1.6cm} tftb\_window}

\vspace*{-.4cm}
\hspace*{-1.6cm}\rule[0in]{16.5cm}{.02cm}
\vspace*{.2cm}

{\bf \large \sf Purpose}\\
\hspace*{1.5cm}
\begin{minipage}[t]{13.5cm}
Window generation.
\end{minipage}
\vspace*{.5cm}

{\bf \large \sf Synopsis}\\
\hspace*{1.5cm}
\begin{minipage}[t]{13.5cm}
\begin{verbatim}
h = tftb\_window(N)
h = tftb\_window(N,name)
h = tftb\_window(N,name,param)
h = tftb\_window(N,name,param,param2)
\end{verbatim}
\end{minipage}
\vspace*{.5cm}

{\bf \large \sf Description}\\
\hspace*{1.5cm}
\begin{minipage}[t]{13.5cm}
        {\ty tftb\_window} yields a window of length {\ty N} with a given shape.\\

\hspace*{-.5cm}\begin{tabular*}{14cm}{p{1.5cm} p{8.5cm} c}
Name & Description & Default value\\
\hline
        {\ty N}      & length of the window\\
        {\ty name}   & name of the window shape & {\ty 'Hamming'}\\
        {\ty param}  & optional parameter\\
        {\ty param2} & second optional parameter\\
 \hline {\ty h}      & output window\\
\hline
\end{tabular*}
\vspace*{.5cm}

       Possible names are :\\
        {\ty 'Hamming', 'Hanning', 'Nuttall', 'Papoulis', 'Harris',
        'Rect', 'Triang', 'Bartlett', 'BartHann', 'Blackman',
        'Gauss', 'Parzen', 'Kaiser', 'Dolph', 'Hanna', 'Nutbess', 'spline'}\\

        For the gaussian window, an optional parameter {\ty k}
        sets the value at both extremities. The default value is {\ty 0.005}.\\
 
        For the Kaiser-Bessel window, an optional parameter
        sets the scale. The default value is {\ty 3*pi}.\\
 
        For the Spline windows, {\ty h=tftb\_window(N,'spline',nfreq,p)}
        yields a spline weighting function of order {\ty p} and frequency
        bandwidth proportional to {\ty nfreq}.

\end{minipage}
\vspace*{1cm}

{\bf \large \sf Example}
\begin{verbatim}
         h=tftb\_window(256,'Gauss',0.005); 
         plot(h);
\end{verbatim}

%\newpage

{\bf \large \sf See Also}\\
\hspace*{1.5cm}
\begin{minipage}[t]{13.5cm}
\begin{verbatim}
dwindow.
\end{verbatim}
\end{minipage}
\vspace*{.5cm}


{\bf \large \sf Reference}\\
\hspace*{1.5cm}
\begin{minipage}[t]{13.5cm}
[1] F. Harris ``On the Use of Windows for Harmonic Analysis with the
Discrete Fourier Transform'', Proceedings of the IEEE, Vol. 66, pp. 51-83,
1978.\\

[2] A.H. Nuttal, "A Two-Parameter Class of Bessel Weighting 
	Functions for Spectral Analysis or Array Processing", 
	IEEE Trans on ASSP, Vol 31, pp 1309-1311, Oct 1983.\\

[3] Y. Ho Ha, J.A. Pearce, "A New Window and Comparison to
	Standard Windows", Trans IEEE ASSP, Vol 37, No 2, 
	pp 298-300, February 1989.\\

[4] C.S. Burrus, ``Multiband Least Squares FIR Filter Design'',
	Trans IEEE SP, Vol 43, No 2, pp 412-421, February 1995.
\end{minipage}

